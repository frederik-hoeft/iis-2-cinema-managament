\chapter{Introduction}
\label{ch:intro}

This applied research project was conducted as part of an assignment for the course \enquote{Informationsinfrastrukturen 2} at \gls{a:fhdw} Hannover, supervised by Prof. Dr. Harald König.

\section{Objectives of this work}

\todo{generic introduction to the problem + what we're even doing here}

\todo{this is a first draft:}
The goal of this work is to develop a network-based solution for managing core cinema infrastructure, such as cinema halls, movie showtimes, and ticket sales. The solution should be generic and easily adaptable to the individual needs of each cinema. In order to achieve this goal, the following objectives are defined:

\begin{itemize}
    \item Develop a conceptual design for a generic cinema management solution.
    \item Implement a prototype of the solution, which can be used as a proof of concept.
    \item Evaluate the solution and identify areas for improvement.
\end{itemize}

\section{About \glslongtext{a:cms}}

% use acronym version once here!
\gls{a:cms}

\todo{because it's nice to give a name to whatever we're doing here... cause \inlinecode{\gls{the-thing}} is easy}

The cinema thing is currently simply called \gls{g:cms}. Change it if you want.

\section{Structure of this work}

Five additional chapters follow this introduction. \Cref{ch:foundation} establishes a theoretical foundation for the subsequent chapters by providing an introduction to \gls{a:mps} code generation, the Spring Boot \gls{a:api}, and the .NET platform. \Cref{ch:problem-analysis} investigates the different use-cases associated with cinema management, ticket sales, and administration, analyzing potential pitfalls and deriving concrete requirements for the solution. \Cref{ch:concept} presents a conceptual solution to address these issues. This solution is then implemented and evaluated in \cref{ch:impl}. Finally, \cref{ch:conclusion} evaluates and discusses the results of this work, the current state of the project, and potential avenues for future development.