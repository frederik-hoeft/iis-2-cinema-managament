\chapter{Theoretical Foundation}
\label{ch:foundation}

This chapter provides a theoretical foundation for this work and introduces the reader to the concepts of code generation, the \gls{g:spring} \gls{a:api}, and the .NET platform. It will serve as a basis for the conceptual design and implementation of the proposed cinema management framework as well as for the evaluation presented thereafter.

\section{Code Generation}

Code generation plays a significant role in software development by enhancing developer productivity. As a crucial component of \gls{a:mdd}, a software development approach that utilizes models as primary development artifacts, code generation contributes to the overall efficiency and effectiveness of the development process. Compared to conventional software development approaches, \gls{a:mdd} provides a higher level of abstraction and improves code reusability \cite{voelter2017model}. In this context, code generation refers to the process of automatically producing source code or machine code from a higher-level abstract description or model, streamlining the development process and reducing manual implementation efforts. This process usually involves a specific language and platform defined by developers \cite{greenfield2002code}.

Various tools are available for code generation, including \glspl{a:ide}, code generators and templates. Of these, the \glsfull{a:mps} by JetBrains will be discussed in further details due to its relevance for this work \cite{biermann2017model}.

\subsection{MPS}

The \glsfull{a:mps} is an \gls{a:ide} and software development platform developed by JetBrains. By using \gls{a:mps}, software developers can automatically generate source code targeting a variety of languages. It is a \gls{a:mdd} tool operating on the principle that the developer designs a model of their proposed system and generates corresponding source code based on this model. This offers developers an immense simplification, enabling them to focus on modeling and defining the requirements for the application instead of manually writing code. Consequently, this reduces the amount of manual work that is traditionally involved in the development of applications, decreasing the likelihood of errors. In addition, the generated source files can be automatically imported into existing projects, which further accelerates the development process. Overall, \gls{a:mps} assists the developers to work quicker and more efficiently allowing them to focus on the development of core functionalities \cite{voelter2013study}.


\section{RESTful APIs}

\subsection{Spring Boot}

\section{Something about .NET probably}

\subsection{System.CommandLine framework}